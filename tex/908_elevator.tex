\documentclass[12pt]{article}
\usepackage{url, graphicx, epstopdf, xcolor, pagecolor}

% color definitions for different subjects
\definecolor{mechanicscolor}{rgb}{1.00,1.00,1.00}
\definecolor{emcolor}{rgb}{1.00,0.95,0.95}

% page layout
\setlength{\topmargin}{-0.25in}
\setlength{\textheight}{9.5in}
\setlength{\headheight}{0in}
\setlength{\headsep}{0in}
\setlength{\parindent}{0in}
\setlength{\parskip}{0.5\baselineskip}

% header and footer
\newenvironment{pottproblem}{%
  ~\hfill {\LARGE\textbf{Physics On The Toilet}} \hfill ~%
  \vfill\large}{%
  \vfill{\normalsize%
  ~\hfill Comments, questions, suggestions? Email \texttt{<pottnyu@gmail.com>}. \hfill ~}}

% problem formatting
\newcommand{\problemname}{Problem}
\newcounter{problem}

% words
\newcommand{\foreign}[1]{\textsl{#1}}
\newcommand{\vs}{\foreign{vs}}

% math
\newcommand{\dd}{\mathrm{d}}
\newcommand{\e}{\mathrm{e}}

% primary units
\newcommand{\rad}{\mathrm{rad}}
\newcommand{\kg}{\mathrm{kg}}
\newcommand{\m}{\mathrm{m}}
\newcommand{\s}{\mathrm{s}}

% secondary units
\renewcommand{\deg}{\mathrm{deg}}
\newcommand{\km}{\mathrm{km}}
\newcommand{\cm}{\mathrm{cm}}
\newcommand{\mm}{\mathrm{mm}}
\newcommand{\ft}{\mathrm{ft}}
\newcommand{\mi}{\mathrm{mi}}
\newcommand{\AU}{\mathrm{AU}}
\newcommand{\ns}{\mathrm{ns}}
\newcommand{\h}{\mathrm{h}}
\newcommand{\yr}{\mathrm{yr}}
\newcommand{\J}{\mathrm{J}}
\newcommand{\eV}{\mathrm{eV}}
\newcommand{\W}{\mathrm{W}}
\newcommand{\Pa}{\mathrm{Pa}}

% derived units
\newcommand{\mps}{\m\,\s^{-1}}
\newcommand{\mph}{\mi\,\h^{-1}}
\newcommand{\mpss}{\m\,\s^{-2}}

% random stuff
\sloppy\sloppypar\raggedbottom\frenchspacing\thispagestyle{empty}

\begin{document}\begin{pottproblem}
\textbf{Problem 908}

NYC has hot summers, with most buildings running heavy air
conditioning (controlled by a thermostat). To save energy, NYU asks
employees to conserve energy in various ways.
For example: \emph{When possible, please take the stairs, not the elevator.}

\textsl{(a)} How much energy does a typical NYU employee
burn in climbing (say) 9 stories of stairs?
Don't forget to account for the fact that for every kcal of mechanical
energy generated, a healthy, fit human also generates about 7 times
more in metabolic load; that is, you burn some 8 times your
mechanically computed power in food calories. (\emph{Bonus:} Why is there this extra load?)

\textsl{(b)} Now imagine (correctly) that all that metabolic energy
gets dumped into the building atmosphere and needs to get corrected by
the building A/C.
The very finest air-conditioning units (\emph{Do we have those?}) are
efficient; they burn about a BTU
for every BTU of cooling they do (that is, for each BTU taken from the inside,
they dump 2-ish BTU outside).
What load does an employee put on the A/C by climbing this far?

\textsl{(c)} For \emph{extremely deep reasons}, an
elevator can't be perfectly efficient either. Can you think of some of
these reasons? Some are fundamental!
Does an elevator burn
more than 8 times the mechanical work it is doing, and does it
drop that load inside the building A/C system? The ``nicest'' assumption you can make
is that the elevators are always crowded, so you are only
looking at the \emph{marginal} cost of bringing up one more person.

\textsl{(d)} \emph{Bonus points:} Answer the more global question, by
adding in the implicit energy and load costs making
and delivering employees their food energy vs getting elevators their electrical energy.
Does our most sustainable future look like the society depicted in \textsl{Wall-E}?

{\normalsize\emph{Adapted from questions posed by Andrei Gruzinov.}}
\end{pottproblem}\end{document}
