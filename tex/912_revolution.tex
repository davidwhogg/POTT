\documentclass[12pt]{article}
\usepackage{url, graphicx, epstopdf, xcolor, pagecolor}

% color definitions for different subjects
\definecolor{mechanicscolor}{rgb}{1.00,1.00,1.00}
\definecolor{emcolor}{rgb}{1.00,0.95,0.95}

% page layout
\setlength{\topmargin}{-0.25in}
\setlength{\textheight}{9.5in}
\setlength{\headheight}{0in}
\setlength{\headsep}{0in}
\setlength{\parindent}{0in}
\setlength{\parskip}{0.5\baselineskip}

% header and footer
\newenvironment{pottproblem}{%
  ~\hfill {\LARGE\textbf{Physics On The Toilet}} \hfill ~%
  \vfill\large}{%
  \vfill{\normalsize%
  ~\hfill Comments, questions, suggestions? Email \texttt{<pottnyu@gmail.com>}. \hfill ~}}

% problem formatting
\newcommand{\problemname}{Problem}
\newcounter{problem}

% words
\newcommand{\foreign}[1]{\textsl{#1}}
\newcommand{\vs}{\foreign{vs}}

% math
\newcommand{\dd}{\mathrm{d}}
\newcommand{\e}{\mathrm{e}}

% primary units
\newcommand{\rad}{\mathrm{rad}}
\newcommand{\kg}{\mathrm{kg}}
\newcommand{\m}{\mathrm{m}}
\newcommand{\s}{\mathrm{s}}

% secondary units
\renewcommand{\deg}{\mathrm{deg}}
\newcommand{\km}{\mathrm{km}}
\newcommand{\cm}{\mathrm{cm}}
\newcommand{\mm}{\mathrm{mm}}
\newcommand{\ft}{\mathrm{ft}}
\newcommand{\mi}{\mathrm{mi}}
\newcommand{\AU}{\mathrm{AU}}
\newcommand{\ns}{\mathrm{ns}}
\newcommand{\h}{\mathrm{h}}
\newcommand{\yr}{\mathrm{yr}}
\newcommand{\J}{\mathrm{J}}
\newcommand{\eV}{\mathrm{eV}}
\newcommand{\W}{\mathrm{W}}
\newcommand{\Pa}{\mathrm{Pa}}

% derived units
\newcommand{\mps}{\m\,\s^{-1}}
\newcommand{\mph}{\mi\,\h^{-1}}
\newcommand{\mpss}{\m\,\s^{-2}}

% random stuff
\sloppy\sloppypar\raggedbottom\frenchspacing\thispagestyle{empty}

\begin{document}\begin{pottproblem}
\textbf{Problem 912}

It's arguable that the most important moment in (human) intellectual
history was the moment in the 1600s (or even late 1500s) in which
humanity realized that \emph{the Earth goes around the Sun}, and not
the other way around, driven by discoveries by Copernicus, Galileo,
Halley, Newton, and many others.  It's even called (amusingly) the
scientific \emph{revolution}.

In the period 1910--1916, Einstein and contemporaries (many in
mathematics) showed that the laws of gravity can be written in
explicitly coordinate-free form. This result is very strong! It says
that the coordinate system can be modified according to any arbitrary
\emph{diffeomorphism}\footnote{A diffeomorphism is an invertible
function that maps one differentiable curvilinear coordinate system to
another such that both the function and its inverse are
differentiable.} and all observables will remain invariant.

In particular, the coordinate system can have its origin comoving with
the Sun, the Earth, Jupiter, or even your face, and everything
gravitational is nonetheless predicted correctly.


\textsl{(a)}~Doesn't this mean that there is no answer, in general
relativity, to the question \textsl{``Does the Sun go around the Earth or does
the Earth go around the Sun?''}?

\textsl{(b)}~Obviously, Solar System dynamics is \emph{easier to
calculate} in a Sun-centered coordinate system (or even better, a
barycentric system). Is the ``ease of calculation'' a good argument to
answer this question? Is that all that Copernicus meant?\footnote{I
think maybe this really is all that Copernicus meant! But I haven't
read the original manuscript.}

\textsl{(c)}~Can you devise an answer to the question \textsl{``Does the Sun
go around the Earth or does the Earth go around the Sun?''} that makes
reference only to observables that are valid in general relativity?

\end{pottproblem}\end{document}
