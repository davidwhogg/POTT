\documentclass[12pt]{article}
\usepackage{url, graphicx, epstopdf, xcolor, pagecolor}

% color definitions for different subjects
\definecolor{mechanicscolor}{rgb}{1.00,1.00,1.00}
\definecolor{emcolor}{rgb}{1.00,0.95,0.95}

% page layout
\setlength{\topmargin}{-0.25in}
\setlength{\textheight}{9.5in}
\setlength{\headheight}{0in}
\setlength{\headsep}{0in}
\setlength{\parindent}{0in}
\setlength{\parskip}{0.5\baselineskip}

% header and footer
\newenvironment{pottproblem}{%
  ~\hfill {\LARGE\textbf{Physics On The Toilet}} \hfill ~%
  \vfill\large}{%
  \vfill{\normalsize%
  ~\hfill Comments, questions, suggestions? Email \texttt{<pottnyu@gmail.com>}. \hfill ~}}

% problem formatting
\newcommand{\problemname}{Problem}
\newcounter{problem}

% words
\newcommand{\foreign}[1]{\textsl{#1}}
\newcommand{\vs}{\foreign{vs}}

% math
\newcommand{\dd}{\mathrm{d}}
\newcommand{\e}{\mathrm{e}}

% primary units
\newcommand{\rad}{\mathrm{rad}}
\newcommand{\kg}{\mathrm{kg}}
\newcommand{\m}{\mathrm{m}}
\newcommand{\s}{\mathrm{s}}

% secondary units
\renewcommand{\deg}{\mathrm{deg}}
\newcommand{\km}{\mathrm{km}}
\newcommand{\cm}{\mathrm{cm}}
\newcommand{\mm}{\mathrm{mm}}
\newcommand{\ft}{\mathrm{ft}}
\newcommand{\mi}{\mathrm{mi}}
\newcommand{\AU}{\mathrm{AU}}
\newcommand{\ns}{\mathrm{ns}}
\newcommand{\h}{\mathrm{h}}
\newcommand{\yr}{\mathrm{yr}}
\newcommand{\J}{\mathrm{J}}
\newcommand{\eV}{\mathrm{eV}}
\newcommand{\W}{\mathrm{W}}
\newcommand{\Pa}{\mathrm{Pa}}

% derived units
\newcommand{\mps}{\m\,\s^{-1}}
\newcommand{\mph}{\mi\,\h^{-1}}
\newcommand{\mpss}{\m\,\s^{-2}}

% random stuff
\sloppy\sloppypar\raggedbottom\frenchspacing\thispagestyle{empty}

\begin{document}\begin{pottproblem}
\textbf{Problem 405}

Astronomers will tell you that the angular resolution of an ideal telescope is
set by the ratio of the wavelength $\lambda$ of the light being detected
to the diameter $D$ of the telescope aperture. That is, an exceedingly distant
point source will appear (in a well-focused, optically good) telescope with
an angular size (in radians) of roughly $\lambda / D$ (ignoring lots of details
like Airy patterns and so forth).

(a) Show that this result can be seen as a direct consequence of quantum
mechanical uncertainty, or that you can't know the momentum of the photon and
the position of the photon at the same time. \textsl{Hint: The direction of
  a photon is related to its transverse momentum, and the fact that the
  photon entered the telescope is related to its transverse position. Does
  that help?}

(b) How does this all apply to the output of a laser? A laser emits a very
collimated beam of photons. But (because of uncertainty) it can't be perfectly
collimated! What properties of the laser set the angular divergence
of the output beam?

\end{pottproblem}\end{document}
