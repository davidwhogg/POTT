\documentclass[12pt]{article}
\usepackage{url, graphicx, epstopdf}

% page layout
\setlength{\topmargin}{-0.25in}
\setlength{\textheight}{9.5in}
\setlength{\headheight}{0in}
\setlength{\headsep}{0in}
\setlength{\parindent}{0in}
\setlength{\parskip}{0.5\baselineskip}

% header and footer
\newenvironment{pottproblem}{%
  ~\hfill {\LARGE\textbf{Physics On The Toilet}} \hfill ~%
  \vfill\large}{%
  \vfill{\normalsize%
  ~\hfill Comments, questions, suggestions? Email \texttt{<pottnyu@gmail.com>}. \hfill ~}}

% problem formatting
\newcommand{\problemname}{Problem}
\newcounter{problem}

% words
\newcommand{\foreign}[1]{\textsl{#1}}
\newcommand{\vs}{\foreign{vs}}

% math
\newcommand{\dd}{\mathrm{d}}
\newcommand{\e}{\mathrm{e}}

% primary units
\newcommand{\rad}{\mathrm{rad}}
\newcommand{\kg}{\mathrm{kg}}
\newcommand{\m}{\mathrm{m}}
\newcommand{\s}{\mathrm{s}}

% secondary units
\renewcommand{\deg}{\mathrm{deg}}
\newcommand{\km}{\mathrm{km}}
\newcommand{\cm}{\mathrm{cm}}
\newcommand{\mm}{\mathrm{mm}}
\newcommand{\ft}{\mathrm{ft}}
\newcommand{\mi}{\mathrm{mi}}
\newcommand{\AU}{\mathrm{AU}}
\newcommand{\ns}{\mathrm{ns}}
\newcommand{\h}{\mathrm{h}}
\newcommand{\yr}{\mathrm{yr}}
\newcommand{\J}{\mathrm{J}}
\newcommand{\eV}{\mathrm{eV}}
\newcommand{\W}{\mathrm{W}}
\newcommand{\Pa}{\mathrm{Pa}}

% derived units
\newcommand{\mps}{\m\,\s^{-1}}
\newcommand{\mph}{\mi\,\h^{-1}}
\newcommand{\mpss}{\m\,\s^{-2}}

% random stuff
\sloppy\sloppypar\raggedbottom\frenchspacing\thispagestyle{empty}

\begin{document}\begin{pottproblem}
\textbf{Problem 903}

Imagine that you have a very, very rigid, thin, solid, circular ring of mass $M$ and radius $a$.
You spin it at an angular frequency $\omega$ around its axis of symmetry.
What happens as the tangential speed $\omega\,a$ approaches the speed of light $c$?

\textsl{(a)} The most straightforward answer is: The ring contracts, because of length contraction.
If this is true, what is its new radius $a'$ as a function of $\omega$?

\textsl{(b)} But wait: How rigid can a ring be? What is the fastest possible speed of
longitudinal waves in the ring?

\textsl{(c)} The speed of a longitudinal wave depends on the density $\rho$ of the ring and its
bulk modulus $K$. The modulus has units of pressure (it is technically stress per unit dimensionless strain).
What, therefore is the highest bulk modulus $K_\mathrm{max}$ the ring can possibly have?

\textsl{(d)} Imagine that the ring has bulk modulus approaching $K_\mathrm{max}$.
That makes the ring elastic, and therefore stretchy.
How stretchy? Does it shrink more (because of length contraction) or stretch more (because of elasticity)
when spun at a relativistic frequency $\omega$?
That is, will $a'$ be greater or less than $a$ as $\omega$ approaches $c/a$?

\textsl{(e)} Diamond is pretty stiff. But nowhere near stiff enough to be relevant here.
What could conceivably be stiff enough to be a good material out of which to make this ring?
\textsl{Hint:} You might need a topological defect!

\end{pottproblem}\end{document}
