\documentclass[12pt]{article}
\usepackage{url, graphicx, epstopdf}

% page layout
\setlength{\topmargin}{-0.25in}
\setlength{\textheight}{9.5in}
\setlength{\headheight}{0in}
\setlength{\headsep}{0in}
\setlength{\parindent}{0in}
\setlength{\parskip}{0.5\baselineskip}

% header and footer
\newenvironment{pottproblem}{%
  ~\hfill {\LARGE\textbf{Physics On The Toilet}} \hfill ~%
  \vfill\large}{%
  \vfill{\normalsize%
  ~\hfill Comments, questions, suggestions? Email \texttt{<pottnyu@gmail.com>}. \hfill ~}}

% problem formatting
\newcommand{\problemname}{Problem}
\newcounter{problem}

% words
\newcommand{\foreign}[1]{\textsl{#1}}
\newcommand{\vs}{\foreign{vs}}

% math
\newcommand{\dd}{\mathrm{d}}
\newcommand{\e}{\mathrm{e}}

% primary units
\newcommand{\rad}{\mathrm{rad}}
\newcommand{\kg}{\mathrm{kg}}
\newcommand{\m}{\mathrm{m}}
\newcommand{\s}{\mathrm{s}}

% secondary units
\renewcommand{\deg}{\mathrm{deg}}
\newcommand{\km}{\mathrm{km}}
\newcommand{\cm}{\mathrm{cm}}
\newcommand{\mm}{\mathrm{mm}}
\newcommand{\ft}{\mathrm{ft}}
\newcommand{\mi}{\mathrm{mi}}
\newcommand{\AU}{\mathrm{AU}}
\newcommand{\ns}{\mathrm{ns}}
\newcommand{\h}{\mathrm{h}}
\newcommand{\yr}{\mathrm{yr}}
\newcommand{\J}{\mathrm{J}}
\newcommand{\eV}{\mathrm{eV}}
\newcommand{\W}{\mathrm{W}}
\newcommand{\Pa}{\mathrm{Pa}}

% derived units
\newcommand{\mps}{\m\,\s^{-1}}
\newcommand{\mph}{\mi\,\h^{-1}}
\newcommand{\mpss}{\m\,\s^{-2}}

% random stuff
\sloppy\sloppypar\raggedbottom\frenchspacing\thispagestyle{empty}

\begin{document}\begin{pottproblem}
\textbf{Problem 403}

The concept of spin in quantum mechanics is mysterious, but this is
not the only example in physics in which angular momentum can exist in
a non-obvious form.
Consider two concentric circular hoops (or concentric cylinders) of
conductor with equal and opposite charge, so there is a radial
electric field between them.
Place these charged hoops into a uniform magnetic field, such that the field is
oriented perpendicular to the plane of the hoops (or down the axis of
the cylinders).

Is there any way to associate an angular momentum with this system?
The answer is ``yes''!
What is that angular momentum?
It might help to think dimensionally (in terms of units), and maybe also
geometrically (the angular momentum is a pseudo-vector, like the magnetic
field).

Is anything \emph{actually rotating?}
Can you do something simple to this system that will convert the
angular momentum into actual rotation of the hoops?

{\normalsize \emph{Problem contributed by Matt Kleban}}

\end{pottproblem}\end{document}
