\documentclass[12pt]{article}
\usepackage{url, graphicx, epstopdf}

% page layout
\setlength{\topmargin}{-0.25in}
\setlength{\textheight}{9.5in}
\setlength{\headheight}{0in}
\setlength{\headsep}{0in}
\setlength{\parindent}{0in}
\setlength{\parskip}{0.5\baselineskip}

% header and footer
\newenvironment{pottproblem}{%
  ~\hfill {\LARGE\textbf{Physics On The Toilet}} \hfill ~%
  \vfill\large}{%
  \vfill{\normalsize%
  ~\hfill Comments, questions, suggestions? Email \texttt{<pottnyu@gmail.com>}. \hfill ~}}

% problem formatting
\newcommand{\problemname}{Problem}
\newcounter{problem}

% words
\newcommand{\foreign}[1]{\textsl{#1}}
\newcommand{\vs}{\foreign{vs}}

% math
\newcommand{\dd}{\mathrm{d}}
\newcommand{\e}{\mathrm{e}}

% primary units
\newcommand{\rad}{\mathrm{rad}}
\newcommand{\kg}{\mathrm{kg}}
\newcommand{\m}{\mathrm{m}}
\newcommand{\s}{\mathrm{s}}

% secondary units
\renewcommand{\deg}{\mathrm{deg}}
\newcommand{\km}{\mathrm{km}}
\newcommand{\cm}{\mathrm{cm}}
\newcommand{\mm}{\mathrm{mm}}
\newcommand{\ft}{\mathrm{ft}}
\newcommand{\mi}{\mathrm{mi}}
\newcommand{\AU}{\mathrm{AU}}
\newcommand{\ns}{\mathrm{ns}}
\newcommand{\h}{\mathrm{h}}
\newcommand{\yr}{\mathrm{yr}}
\newcommand{\J}{\mathrm{J}}
\newcommand{\eV}{\mathrm{eV}}
\newcommand{\W}{\mathrm{W}}
\newcommand{\Pa}{\mathrm{Pa}}

% derived units
\newcommand{\mps}{\m\,\s^{-1}}
\newcommand{\mph}{\mi\,\h^{-1}}
\newcommand{\mpss}{\m\,\s^{-2}}

% random stuff
\sloppy\sloppypar\raggedbottom\frenchspacing\thispagestyle{empty}

\begin{document}\begin{pottproblem}
\textbf{Problem 201}

A car battery stores a lot of energy!
A typical car battery has a capacity of $50\,\A\,\h$ (amp-hours) at $12\,\V$.

A really stupid thing to do is to short out a car battery with a copper wire!
(Don't try this at home.)
Imagine you short out a car battery like this with a copper wire that has a
length of, say, $0.5\,\m$ and a diameter of, say, $5\,\mm$.
Questions, from easy to hard:

\textsl{(a)}~How much energy is released by this short-out?
\textsl{Extra credit:} Convert your answer into TNT-equivalent-mass units.

\textsl{(b)}~If the resistivity of copper is $1.7\times 10^{-8}\,\ohm\,\m$, how
long does the battery take to discharge? Your simple answer is wrong here, because of
course the battery itself has an internal impedance. What do you think is a good
setting for the internal impedance of a car battery and why?

\textsl{(c)}~If copper contains something like one free electron per atom
(we here at POTT headquarters have no idea if this is true), how far
does a typical electron move along the copper wire
during the process of fully shorting out this battery?

\end{pottproblem}\end{document}
