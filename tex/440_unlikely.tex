\documentclass[12pt]{article}
\usepackage{url, graphicx, epstopdf}

% page layout
\setlength{\topmargin}{-0.25in}
\setlength{\textheight}{9.5in}
\setlength{\headheight}{0in}
\setlength{\headsep}{0in}
\setlength{\parindent}{0in}
\setlength{\parskip}{0.5\baselineskip}

% header and footer
\newenvironment{pottproblem}{%
  ~\hfill {\LARGE\textbf{Physics On The Toilet}} \hfill ~%
  \vfill\large}{%
  \vfill{\normalsize%
  ~\hfill Comments, questions, suggestions? Email \texttt{<pottnyu@gmail.com>}. \hfill ~}}

% problem formatting
\newcommand{\problemname}{Problem}
\newcounter{problem}

% words
\newcommand{\foreign}[1]{\textsl{#1}}
\newcommand{\vs}{\foreign{vs}}

% math
\newcommand{\dd}{\mathrm{d}}
\newcommand{\e}{\mathrm{e}}

% primary units
\newcommand{\rad}{\mathrm{rad}}
\newcommand{\kg}{\mathrm{kg}}
\newcommand{\m}{\mathrm{m}}
\newcommand{\s}{\mathrm{s}}

% secondary units
\renewcommand{\deg}{\mathrm{deg}}
\newcommand{\km}{\mathrm{km}}
\newcommand{\cm}{\mathrm{cm}}
\newcommand{\mm}{\mathrm{mm}}
\newcommand{\ft}{\mathrm{ft}}
\newcommand{\mi}{\mathrm{mi}}
\newcommand{\AU}{\mathrm{AU}}
\newcommand{\ns}{\mathrm{ns}}
\newcommand{\h}{\mathrm{h}}
\newcommand{\yr}{\mathrm{yr}}
\newcommand{\J}{\mathrm{J}}
\newcommand{\eV}{\mathrm{eV}}
\newcommand{\W}{\mathrm{W}}
\newcommand{\Pa}{\mathrm{Pa}}

% derived units
\newcommand{\mps}{\m\,\s^{-1}}
\newcommand{\mph}{\mi\,\h^{-1}}
\newcommand{\mpss}{\m\,\s^{-2}}

% random stuff
\sloppy\sloppypar\raggedbottom\frenchspacing\thispagestyle{empty}

\begin{document}\begin{pottproblem}
\textbf{Problem 410}

Many things are possible (if very unlikely) in quantum mechanics that
are impossible classically (tunneling through a potential energy
barrier, for instance).

\textsl{(1)}~Is there anything that is completely \emph{impossible} in
quantum mechanics? 

\textsl{(2)}~Of the processes that are \emph{not} impossible---and for
which you can calculate (or estimate) the probability---which one can
you think of that has the lowest probability?

\textsl{(3)} Do extremely small probabilities have any meaning? If
not, what is the smallest probability that we can assign meaning to?
Does this change your answer to part \textsl{(1)}?

{\normalsize\emph{Thanks to Matt Kleban for contributing this problem.}}
\end{pottproblem}\end{document}
