\documentclass[12pt]{article}
\usepackage{url, graphicx, epstopdf}

% page layout
\setlength{\topmargin}{-0.25in}
\setlength{\textheight}{9.5in}
\setlength{\headheight}{0in}
\setlength{\headsep}{0in}
\setlength{\parindent}{0in}
\setlength{\parskip}{0.5\baselineskip}

% header and footer
\newenvironment{pottproblem}{%
  ~\hfill {\LARGE\textbf{Physics On The Toilet}} \hfill ~%
  \vfill\large}{%
  \vfill{\normalsize%
  ~\hfill Comments, questions, suggestions? Email \texttt{<pottnyu@gmail.com>}. \hfill ~}}

% problem formatting
\newcommand{\problemname}{Problem}
\newcounter{problem}

% words
\newcommand{\foreign}[1]{\textsl{#1}}
\newcommand{\vs}{\foreign{vs}}

% math
\newcommand{\dd}{\mathrm{d}}
\newcommand{\e}{\mathrm{e}}

% primary units
\newcommand{\rad}{\mathrm{rad}}
\newcommand{\kg}{\mathrm{kg}}
\newcommand{\m}{\mathrm{m}}
\newcommand{\s}{\mathrm{s}}

% secondary units
\renewcommand{\deg}{\mathrm{deg}}
\newcommand{\km}{\mathrm{km}}
\newcommand{\cm}{\mathrm{cm}}
\newcommand{\mm}{\mathrm{mm}}
\newcommand{\ft}{\mathrm{ft}}
\newcommand{\mi}{\mathrm{mi}}
\newcommand{\AU}{\mathrm{AU}}
\newcommand{\ns}{\mathrm{ns}}
\newcommand{\h}{\mathrm{h}}
\newcommand{\yr}{\mathrm{yr}}
\newcommand{\J}{\mathrm{J}}
\newcommand{\eV}{\mathrm{eV}}
\newcommand{\W}{\mathrm{W}}
\newcommand{\Pa}{\mathrm{Pa}}

% derived units
\newcommand{\mps}{\m\,\s^{-1}}
\newcommand{\mph}{\mi\,\h^{-1}}
\newcommand{\mpss}{\m\,\s^{-2}}

% random stuff
\sloppy\sloppypar\raggedbottom\frenchspacing\thispagestyle{empty}

\begin{document}\begin{pottproblem}
\textbf{Problem 930}

The principle of equivalence is that the gravitational charge (the
thing that sets the magnitude of the gravitational force) is identical
to the inertial mass (the $m$ in $F=m\,a$). General relativity (for
example) flows from this principle (plus other considerations).

\textsl{(a)}~The principle of equivalence is also sometimes stated as
that an observer (in a closed box, say) can't distinguish between
being in a non-accelerating box in a uniform gravitational field and
being in an accelerating box in a vanishing gravitational field. How
is this statement of the principle \emph{the same} as the one about
mass and charge?

\textsl{(b)}~As soon as Einstein (re-)formulated the principle, but
before he figured out general relativity (which is hard!), he could
deduce that the principle---if taken to be fundamental---will lead to
the consequence that there will be a gravitational redshift, the
consequence that light rays will bend in (be lensed by) a
gravitational field, and the consequence that clocks at different
gravitational potentials will tick at different rates. How do these
three consequences flow so directly from the principle?

\textsl{(c)}~Does the existence of black holes follow directly from
the principle? And if so, what properties of black holes can be
deduced from the principle? Or does the analysis of black holes
require a full specification of general relativity?

\end{pottproblem}\end{document}
