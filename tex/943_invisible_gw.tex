\documentclass[12pt]{article}
\usepackage{url, graphicx, epstopdf, xcolor, pagecolor}

% color definitions for different subjects
\definecolor{mechanicscolor}{rgb}{1.00,1.00,1.00}
\definecolor{emcolor}{rgb}{1.00,0.95,0.95}

% page layout
\setlength{\topmargin}{-0.25in}
\setlength{\textheight}{9.5in}
\setlength{\headheight}{0in}
\setlength{\headsep}{0in}
\setlength{\parindent}{0in}
\setlength{\parskip}{0.5\baselineskip}

% header and footer
\newenvironment{pottproblem}{%
  ~\hfill {\LARGE\textbf{Physics On The Toilet}} \hfill ~%
  \vfill\large}{%
  \vfill{\normalsize%
  ~\hfill Comments, questions, suggestions? Email \texttt{<pottnyu@gmail.com>}. \hfill ~}}

% problem formatting
\newcommand{\problemname}{Problem}
\newcounter{problem}

% words
\newcommand{\foreign}[1]{\textsl{#1}}
\newcommand{\vs}{\foreign{vs}}

% math
\newcommand{\dd}{\mathrm{d}}
\newcommand{\e}{\mathrm{e}}

% primary units
\newcommand{\rad}{\mathrm{rad}}
\newcommand{\kg}{\mathrm{kg}}
\newcommand{\m}{\mathrm{m}}
\newcommand{\s}{\mathrm{s}}

% secondary units
\renewcommand{\deg}{\mathrm{deg}}
\newcommand{\km}{\mathrm{km}}
\newcommand{\cm}{\mathrm{cm}}
\newcommand{\mm}{\mathrm{mm}}
\newcommand{\ft}{\mathrm{ft}}
\newcommand{\mi}{\mathrm{mi}}
\newcommand{\AU}{\mathrm{AU}}
\newcommand{\ns}{\mathrm{ns}}
\newcommand{\h}{\mathrm{h}}
\newcommand{\yr}{\mathrm{yr}}
\newcommand{\J}{\mathrm{J}}
\newcommand{\eV}{\mathrm{eV}}
\newcommand{\W}{\mathrm{W}}
\newcommand{\Pa}{\mathrm{Pa}}

% derived units
\newcommand{\mps}{\m\,\s^{-1}}
\newcommand{\mph}{\mi\,\h^{-1}}
\newcommand{\mpss}{\m\,\s^{-2}}

% random stuff
\sloppy\sloppypar\raggedbottom\frenchspacing\thispagestyle{empty}

\begin{document}\begin{pottproblem}
\textbf{Problem 943}

\textsl{(a)}
Consider an \textsf{L}-shaped gravitational-wave antenna, like the Hanford detector
of \textsl{LIGO}.
Are there black-hole inspirals to which this antenna is completely insensitive,
even in the absence of noise?
That is, what are the inspiral events that are fundamentally and formally invisible to this
detector?
Assume that the event is far, far shorter than one day, so Earth rotation can be ignored.
\emph{Note: You don't need to know much about General Relativity in order to answer this question; use symmetries!}

\textsl{(b)}
Are your answers still valid if the black holes have spin?
What spin directions can the black holes have that will invalidate
or keep valid your answer to part (a)?

\textsl{(c)}
Now let the Earth rotate through a finite angle during the event.
Are there still formally invisible events?

\textsl{(d)}
Now generalize part (a) to electromagnetism, where the source is a 
spinning electric dipole, and the antenna is a straight quarter-wave antenna.

\end{pottproblem}\end{document}
