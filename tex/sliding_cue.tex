\documentclass[12pt]{article}
\usepackage{url, graphicx, epstopdf, xcolor, pagecolor}

% color definitions for different subjects
\definecolor{mechanicscolor}{rgb}{1.00,1.00,1.00}
\definecolor{emcolor}{rgb}{1.00,0.95,0.95}

% page layout
\setlength{\topmargin}{-0.25in}
\setlength{\textheight}{9.5in}
\setlength{\headheight}{0in}
\setlength{\headsep}{0in}
\setlength{\parindent}{0in}
\setlength{\parskip}{0.5\baselineskip}

% header and footer
\newenvironment{pottproblem}{%
  ~\hfill {\LARGE\textbf{Physics On The Toilet}} \hfill ~%
  \vfill\large}{%
  \vfill{\normalsize%
  ~\hfill Comments, questions, suggestions? Email \texttt{<pottnyu@gmail.com>}. \hfill ~}}

% problem formatting
\newcommand{\problemname}{Problem}
\newcounter{problem}

% words
\newcommand{\foreign}[1]{\textsl{#1}}
\newcommand{\vs}{\foreign{vs}}

% math
\newcommand{\dd}{\mathrm{d}}
\newcommand{\e}{\mathrm{e}}

% primary units
\newcommand{\rad}{\mathrm{rad}}
\newcommand{\kg}{\mathrm{kg}}
\newcommand{\m}{\mathrm{m}}
\newcommand{\s}{\mathrm{s}}

% secondary units
\renewcommand{\deg}{\mathrm{deg}}
\newcommand{\km}{\mathrm{km}}
\newcommand{\cm}{\mathrm{cm}}
\newcommand{\mm}{\mathrm{mm}}
\newcommand{\ft}{\mathrm{ft}}
\newcommand{\mi}{\mathrm{mi}}
\newcommand{\AU}{\mathrm{AU}}
\newcommand{\ns}{\mathrm{ns}}
\newcommand{\h}{\mathrm{h}}
\newcommand{\yr}{\mathrm{yr}}
\newcommand{\J}{\mathrm{J}}
\newcommand{\eV}{\mathrm{eV}}
\newcommand{\W}{\mathrm{W}}
\newcommand{\Pa}{\mathrm{Pa}}

% derived units
\newcommand{\mps}{\m\,\s^{-1}}
\newcommand{\mph}{\mi\,\h^{-1}}
\newcommand{\mpss}{\m\,\s^{-2}}

% random stuff
\sloppy\sloppypar\raggedbottom\frenchspacing\thispagestyle{empty}

\begin{document}
\pottheader

\begin{list}{}{}
\item{(a)}~Immediately after being hit, at $t=0$, a cue ball of mass
$M$ and radius $R$ slides along the pool-table felt at speed $v_i$, not rotating
at all.  As time goes on, the ball slows down (because of friction)
and, at the same time, starts to spin.  Draw a free-body diagram for
the cue ball.

\item{(b)}~At what time $t_\mathrm{r}$ does the ball get to the
situation of ``rolling without slipping''?  Assume that there is a
coefficient $\mu$ of sliding friction. You will have to look up (or
compute) the moment of inertia $I$ for a uniform sphere.

\item{(c)}~Plot $v(t)$ and $R\,\omega(t)$ vs $t$ on a single plot.
\emph{Note that the two things have the same
dimensions.}  Clearly label $t_\mathrm{r}$ on your diagram with a
line.
Interpret all the polygonal areas you can identify on that plot.

\item{(d)}~In the long run, the cue ball will stop rolling. Explain
why or how. \emph{Note that there are many qualitatively different
answers for a problem that contains the words ``explain why''.
Provide as many as you can.}
\end{list}

\pottfooter
\end{document}
