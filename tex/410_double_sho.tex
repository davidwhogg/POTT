\documentclass[12pt]{article}
\usepackage{url, graphicx, epstopdf}

% page layout
\setlength{\topmargin}{-0.25in}
\setlength{\textheight}{9.5in}
\setlength{\headheight}{0in}
\setlength{\headsep}{0in}
\setlength{\parindent}{0in}
\setlength{\parskip}{0.5\baselineskip}

% header and footer
\newenvironment{pottproblem}{%
  ~\hfill {\LARGE\textbf{Physics On The Toilet}} \hfill ~%
  \vfill\large}{%
  \vfill{\normalsize%
  ~\hfill Comments, questions, suggestions? Email \texttt{<pottnyu@gmail.com>}. \hfill ~}}

% problem formatting
\newcommand{\problemname}{Problem}
\newcounter{problem}

% words
\newcommand{\foreign}[1]{\textsl{#1}}
\newcommand{\vs}{\foreign{vs}}

% math
\newcommand{\dd}{\mathrm{d}}
\newcommand{\e}{\mathrm{e}}

% primary units
\newcommand{\rad}{\mathrm{rad}}
\newcommand{\kg}{\mathrm{kg}}
\newcommand{\m}{\mathrm{m}}
\newcommand{\s}{\mathrm{s}}

% secondary units
\renewcommand{\deg}{\mathrm{deg}}
\newcommand{\km}{\mathrm{km}}
\newcommand{\cm}{\mathrm{cm}}
\newcommand{\mm}{\mathrm{mm}}
\newcommand{\ft}{\mathrm{ft}}
\newcommand{\mi}{\mathrm{mi}}
\newcommand{\AU}{\mathrm{AU}}
\newcommand{\ns}{\mathrm{ns}}
\newcommand{\h}{\mathrm{h}}
\newcommand{\yr}{\mathrm{yr}}
\newcommand{\J}{\mathrm{J}}
\newcommand{\eV}{\mathrm{eV}}
\newcommand{\W}{\mathrm{W}}
\newcommand{\Pa}{\mathrm{Pa}}

% derived units
\newcommand{\mps}{\m\,\s^{-1}}
\newcommand{\mph}{\mi\,\h^{-1}}
\newcommand{\mpss}{\m\,\s^{-2}}

% random stuff
\sloppy\sloppypar\raggedbottom\frenchspacing\thispagestyle{empty}

\begin{document}\begin{pottproblem}
\textbf{Problem 410}

Consider a symmetric double-well potential in one-dimensional non-relativistic
quantum mechanics.  Each well is very similar, and approximately quadratic, and the
barrier between them is thick and high, so that $S_{\mathrm WKB} \gg
\hbar$.

Create an initial position wavefunction that is localized
in one well; for instance, a Gaussian centered on one well with a
width corresponding to the would-be ground state of that one well.
Show or remind yourself
that the the square of this state will oscillate back and forth
between the wells with a period that is exponential in $S_{\mathrm WKB}$.
So, after one half-period there will be almost zero probability to
find the particle in the original well, and almost unit probability to find it in
the other.

Now consider an asymmetric double well, similar to the above, except
with a shift $\Delta V$ in the potential energy of one minimum with
respect to the other.  What is the condition on $\Delta V$ such that
the same initial wavefunction will oscillate back and forth in this
same way?  If $\Delta V$ does not satisfy this condition, describe the
time evolution qualitatively.

{\normalsize\emph{Thanks to Matt Kleban for contributing this problem.}}
\end{pottproblem}\end{document}
