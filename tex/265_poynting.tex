\documentclass[12pt]{article}
\usepackage{url, graphicx, epstopdf}

% page layout
\setlength{\topmargin}{-0.25in}
\setlength{\textheight}{9.5in}
\setlength{\headheight}{0in}
\setlength{\headsep}{0in}
\setlength{\parindent}{0in}
\setlength{\parskip}{0.5\baselineskip}

% header and footer
\newenvironment{pottproblem}{%
  ~\hfill {\LARGE\textbf{Physics On The Toilet}} \hfill ~%
  \vfill\large}{%
  \vfill{\normalsize%
  ~\hfill Comments, questions, suggestions? Email \texttt{<pottnyu@gmail.com>}. \hfill ~}}

% problem formatting
\newcommand{\problemname}{Problem}
\newcounter{problem}

% words
\newcommand{\foreign}[1]{\textsl{#1}}
\newcommand{\vs}{\foreign{vs}}

% math
\newcommand{\dd}{\mathrm{d}}
\newcommand{\e}{\mathrm{e}}

% primary units
\newcommand{\rad}{\mathrm{rad}}
\newcommand{\kg}{\mathrm{kg}}
\newcommand{\m}{\mathrm{m}}
\newcommand{\s}{\mathrm{s}}

% secondary units
\renewcommand{\deg}{\mathrm{deg}}
\newcommand{\km}{\mathrm{km}}
\newcommand{\cm}{\mathrm{cm}}
\newcommand{\mm}{\mathrm{mm}}
\newcommand{\ft}{\mathrm{ft}}
\newcommand{\mi}{\mathrm{mi}}
\newcommand{\AU}{\mathrm{AU}}
\newcommand{\ns}{\mathrm{ns}}
\newcommand{\h}{\mathrm{h}}
\newcommand{\yr}{\mathrm{yr}}
\newcommand{\J}{\mathrm{J}}
\newcommand{\eV}{\mathrm{eV}}
\newcommand{\W}{\mathrm{W}}
\newcommand{\Pa}{\mathrm{Pa}}

% derived units
\newcommand{\mps}{\m\,\s^{-1}}
\newcommand{\mph}{\mi\,\h^{-1}}
\newcommand{\mpss}{\m\,\s^{-2}}

% random stuff
\sloppy\sloppypar\raggedbottom\frenchspacing\thispagestyle{empty}

\begin{document}\begin{pottproblem}
\textbf{Problem 265}

Here's a (supposedly) elementary problem for which different faculty
members in the NYU Physics Department disagree on the answer:

Imagine a circuit that contains a battery providing voltage $V$ and a
resistor of resistance $R$, connected by very highly conducting
(effectively zero resistance) wires.  In this circuit, a current
$I=V/R$ flows steadily, and a power (energy per unit time)
$I\,V=V^2/R$ is steadily dissipated in the resistor.

The question is: How does the energy flow from the battery to the
resistor? There can't be action at a distance, so the answer is
expected to involve the electric and magnetic fields. But remember: In
those conducting wires, the electric field is exactly zero! Does the
energy flow through the space between the battery and the resistor?

One calculation one can do is to compute the Poynting flux
$$
\vec{S} = \frac{1}{\mu_0}\,\vec{E}\times\vec{B}
$$
at the surface of the resistor, and integrate it over the surface of
the resistor.  Recall that there is an electric field inside a
resistor, and there is a magnetic field when current is flowing.  When
you do this calculation for a cylindrical resistor of resistivity
$\rho$ do you get the correct total power flow?

If you did, you might think that the problem is solved. But if the
power is really flowing in those fields, what experiment could you
design to measure or capture or intercept the power flowing in those
fields?

{\normalsize\emph{Thanks to Matt Kleban (NYU) and Shura Grosberg (NYU) for input!}}
\end{pottproblem}\end{document}
