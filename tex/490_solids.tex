\documentclass[12pt]{article}
\usepackage{url, graphicx, epstopdf, xcolor, pagecolor}

% color definitions for different subjects
\definecolor{mechanicscolor}{rgb}{1.00,1.00,1.00}
\definecolor{emcolor}{rgb}{1.00,0.95,0.95}

% page layout
\setlength{\topmargin}{-0.25in}
\setlength{\textheight}{9.5in}
\setlength{\headheight}{0in}
\setlength{\headsep}{0in}
\setlength{\parindent}{0in}
\setlength{\parskip}{0.5\baselineskip}

% header and footer
\newenvironment{pottproblem}{%
  ~\hfill {\LARGE\textbf{Physics On The Toilet}} \hfill ~%
  \vfill\large}{%
  \vfill{\normalsize%
  ~\hfill Comments, questions, suggestions? Email \texttt{<pottnyu@gmail.com>}. \hfill ~}}

% problem formatting
\newcommand{\problemname}{Problem}
\newcounter{problem}

% words
\newcommand{\foreign}[1]{\textsl{#1}}
\newcommand{\vs}{\foreign{vs}}

% math
\newcommand{\dd}{\mathrm{d}}
\newcommand{\e}{\mathrm{e}}

% primary units
\newcommand{\rad}{\mathrm{rad}}
\newcommand{\kg}{\mathrm{kg}}
\newcommand{\m}{\mathrm{m}}
\newcommand{\s}{\mathrm{s}}

% secondary units
\renewcommand{\deg}{\mathrm{deg}}
\newcommand{\km}{\mathrm{km}}
\newcommand{\cm}{\mathrm{cm}}
\newcommand{\mm}{\mathrm{mm}}
\newcommand{\ft}{\mathrm{ft}}
\newcommand{\mi}{\mathrm{mi}}
\newcommand{\AU}{\mathrm{AU}}
\newcommand{\ns}{\mathrm{ns}}
\newcommand{\h}{\mathrm{h}}
\newcommand{\yr}{\mathrm{yr}}
\newcommand{\J}{\mathrm{J}}
\newcommand{\eV}{\mathrm{eV}}
\newcommand{\W}{\mathrm{W}}
\newcommand{\Pa}{\mathrm{Pa}}

% derived units
\newcommand{\mps}{\m\,\s^{-1}}
\newcommand{\mph}{\mi\,\h^{-1}}
\newcommand{\mpss}{\m\,\s^{-2}}

% random stuff
\sloppy\sloppypar\raggedbottom\frenchspacing\thispagestyle{empty}

\begin{document}\begin{pottproblem}
\textbf{Problem 490}

The particles in an atomic lattice make up a vanishingly small fraction
of the total volume of the solid.
In this sense, normal materials are almost entirely ``empty space''.
Of course this is misleading:
When you put a mug down on a table, the mug doesn't pass through the table!
What causes the repulsion that keeps the mug and table from inter-penetrating?

\textsl{(a)} Many people think that the repulsion is classical electrostatic repulsion.
Is this possible?
Think about two incommensurate electrostatic crystals made up of $+e$ and $-e$ charges
in two rigid, globally neutral lattices.
Show (at any level of rigor you like) that these two crystals will in fact \emph{attract}
one another, not repel, and they will have lowest energy when fully inter-penetrated.

\textsl{(b)} Show that making the crystals non-rigid makes this
problem even worse!

\textsl{(c)} If all this is true, what makes solids repel one another when they come into contact?

\textsl{(d)} Historically, did Maxwell and contemporaries understand this argument?
\textit{Maybe that's a question for the internet, not the toilet!}

\end{pottproblem}\end{document}
